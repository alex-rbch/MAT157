%%%%%%%%%%%%%%%%%%%%%%%%%%%%%%%%%%%%%%%%%%%%%%%%%%%%%%%%%%%%%%%%%%%%%%%%%%%%%%%%%%%%
\documentclass{article}
\usepackage[margin=1in]{geometry} 
\usepackage{amsmath,amsthm,amssymb,amsfonts, fancyhdr, color, comment, graphicx, environ}
\usepackage{xcolor}
\usepackage{multicol}
\usepackage{mdframed}
\usepackage[shortlabels]{enumitem}
\usepackage{indentfirst}
\usepackage{hyperref}
\usepackage{mathtools} % was added on 2021-09-24
\hypersetup{
    colorlinks=true,
    linkcolor=blue,
    filecolor=magenta,      
    urlcolor=blue,
}

\pagestyle{fancy}

\newenvironment{problem}[2][Problem]
    { \begin{mdframed}[backgroundcolor=gray!20] \textbf{#1 #2}\\}
    {  \end{mdframed}}

\newenvironment{impdef}[1]
    { \begin{mdframed}[backgroundcolor=green!10] \textbf{#1} \vspace{0.15cm}\\}
    {  \end{mdframed}}

\newenvironment{regdef}[2][Def.]
    { \begin{mdframed}[backgroundcolor=yellow!10] \textbf{#1} {#2}. \vspace{0.05cm}\\}
    {  \end{mdframed}}
    
\newenvironment{theorem}[2][Thm.]
    { \begin{mdframed}[backgroundcolor=blue!10] \textbf{#1} {#2} \vspace{0.2cm}\\}
    {  \end{mdframed}}

%%%%%%%%%%%%%%%%%%%%%%%%%%%%%%%%%%%%%%%%%%%%%
% Define solution environment
\newenvironment{solution}[2][Solution.]
    { \begin{mdframed}[] \textbf{#1 #2} \\}
    {  \end{mdframed}}    

% Define proof environment
\newenvironment{prf}[2][Proof.]
    { \begin{mdframed}[] \textbf{#1 #2} \\}
    {  \end{mdframed}}
%%%%%%%%%%%%%%%%%%%%%%%%%%%%%%%%%%%%%%%%%%%%%
%Fill in the appropriate information below
\lhead{}
\rhead{MAT157: Alex R} 
\chead{\textbf{W1 Properties and order relations on $\R$}}
%%%%%%%%%%%%%%%%%%%%%%%%%%%%%%%%%%%%%%%%%%%%%

% my commands, ignore these if you want
\newcommand{\N}{\mathbb{N}}
\newcommand{\Z}{\mathbb{Z}}
\newcommand{\Q}{\mathbb{Q}}
\newcommand{\R}{\mathbb{R}}
\newcommand{\C}{\mathbb{C}}
\newcommand{\T}{\mathcal{T}}
\newcommand{\pf}{\textit{Proof. }}
\newcommand{\lm}{\textit{Lemma. }}
\newcommand{\union}{\cup}
\newcommand{\inter}{\cap}
\newcommand{\ep}{\hfill $\blacksquare$\nl}
\newcommand{\ec}{\hfill $\square$\nn}


\begin{document}
    
    \begin{impdef}{FUNCTION}
        If $A$ and $B$ are sets, a \textit{function} is a unique assignment of every element in $A$ to an element of $B$. We write $f: A \to B$ to denote a function.
        \begin{itemize}
            \item $\textbf{Domain} = A$
            \item $\textbf{Codomain} = B$
            \item $\textbf{Range} = f(A) = \{f(x): x \in A\} \subseteq B$
        \end{itemize}
    \end{impdef}
        
    \begin{regdef}{Well-defined function}
        A map $f: A \to B$ is \textit{well-defined}, if $(\forall x, y \in A)[x=y \Rightarrow f(x)=f(y)]$.
    \end{regdef}
    
    \begin{impdef}{INJECTIVITY}
        A function $f: A \to B$ is \textit{injective}, if $(\forall x, y \in A)[f(x)=f(y) \Rightarrow x=y]$.
        \begin{itemize}
            \item[\textbf{--}] Every $A$ has unique $B$.
            \item[\textbf{--}] Every $B$ comes from unique $A$ or nothing at all.
        \end{itemize}
    \end{impdef}
    
    \begin{impdef}{SURJECTIVITY}
        A function $f: A \to B$ is \textit{surjective}, if $(\forall y \in B)[\exists x \in A, f(x) = y]$.
        \begin{itemize}
            \item[\textbf{--}] Every element in the codomain is obtainable.
        \end{itemize}
    \end{impdef}
    
    \begin{impdef}{BIJECTIVITY}
        A function $f: A \to B$ is \textit{bijective}, if it is both \textit{injective} and \textit{surjective}.
        \begin{itemize}
            \item[\textbf{--}] Every element is obtainable in a unique way.
            \item[\textbf{--}] There is matching between elements of $A$ and $B$, i.e. function $f$ is \textit{invertible}. 
        \end{itemize}
    \end{impdef}
    
    \begin{regdef}{Cardinality}
        \textit{Cardinality} of a set $S$ is the a measure of the "number of elements" in $S$, denoted $|S|$.
        \begin{itemize}
            \item \textbf{Injectivity.} If there is an injection from $S\to T$, then $|S| \le |T|$.
            \item \textbf{Countability.} A set $S$ is \textit{countable}, if $|S| \le |\N|$.
            \item \textbf{Bijectivity.} $|S| = |T|$, if there exists a bijection $S\to T$.
        \end{itemize}
    \end{regdef}
    
    \begin{theorem}{\textbf{[Cantor–Bernstein-Schröder]}, a.k.a. the \textbf{CBT}}
        If $S$, $T$ are two sets with $|S| \le |T|$ and $|T| \le |S|$, then $|S| = |T|$.
        \begin{itemize}
            \item[\textbf{--}] "Left injection" + "Right injection" = "Bijection".
        \end{itemize}
    \end{theorem}
    
    \clearpage
    
    \begin{theorem}{Transitivity of injectivity.}
        If $f: B\to C$ and $g: A\to B$ are both injective then their composition $f \circ g$ is also injective.
    \end{theorem}
    \begin{prf}{}
        By the definition of injectivity, we have the following.
        \[(\forall a_1, a_2)[g(a_1) = g(a_2) \Rightarrow a_1 = a_2]\]
        \[(\forall b_1, b_2)[f(b_1) = f(b_2) \Rightarrow b_1 = b_2]\]
        Want to show, the following.
        
        \[(\forall a_1, a_2)[f(g(a_1)) = f(g(a_2)) \Rightarrow a_1 = a_2]\]
        
        Assume that $f(g(a_1)) = f(g(a_2))$. 
        \begin{itemize}
            \item Since $f$ is injective, $g(a_1) = g(a_2)$. 
            \item Since $g$ is injective, $a_1 = a_2$.
        \end{itemize} \ep
    \end{prf}
    
    \begin{problem}{1.}
        If $f: B \to C$ and $g: A\to B$ are such that $f \circ g$ is injective, then $g$ is injective. Also, $f$ does not have to be injective.
    \end{problem}
    
    \begin{solution}{}
        We have the following.
        \[(\forall a, b)[f(g(a)) = f(g(b)) \Rightarrow a = b]\]
        Want to show the following.
        \[(\forall a, b)[g(a) = g(b) \Rightarrow a = b]\]
        
        Assume $g(a) = g(b)$.
        \begin{itemize}
            \item Since $f$ is a well-defined function, $f(g(a)) = f(g(b))$.
            \item Since $f \circ g$ is injective, $a = b$.
        \end{itemize} \ec
    \end{solution}
    
    \clearpage
    
    \begin{impdef}{INVERTIBILITY}
        \textit{Inverse} of a function $f:A\to B$ is a function $f^{-1}:B\to A$ that satisfies $f^{-1} \circ f = \text{id}_A$ and $f \circ f^{-1} = \text{id}_B$.
        
        \begin{itemize}
            \item Function is called \textit{left-invertible}, if $f^{-1} \circ f = \text{id}_A$.
            \item Function is called \textit{right-invertible}, if $f\circ f^{-1} = \text{id}_B$.
        \end{itemize}
    \end{impdef}
    
    \begin{theorem}{Injectivity $\Leftrightarrow$ Left-invertibility}
        Suppose $g: A\to B$ is a function. Then $g$ is injective if and only if there exists a function $h: B\to A$ such that $(h\circ g) = \text{id}_A$.
    \end{theorem}
    
    \begin{prf}{($\Longleftarrow$), (mimics Problem 1.)}
        We have that $h \circ g = \text{id}_A$, consequently the following is true.
        \[(\forall a,b \in A)[h(g(a)) = h(g(b)) \Rightarrow a = b]\]
        Want to show the following.
        \[(\forall a,b \in A)[g(a) = g(b) \Rightarrow a = b]\]
        
        Find the exact solution in \textbf{Problem 1.}
    \end{prf}
    
    \begin{prf}{($\Longrightarrow$)}
        The function can be defined explicitly.
        \begin{enumerate}
            \item Fix some element $a_0 \in A$.
            \item Define $h: B\to A$ as
            $\left[h(b) = \begin{cases}
                \hfill
                a & \text{if $g(a) = b$}\\
                a_0 & \text{otherwise}
            \end{cases}\right]$.
        \end{enumerate}
        \[(h\circ g)(a) = h(g(a)) = a, \hspace{0,5cm}\text{so we have}\hspace{0,5cm} h\circ g = \text{id}_A\]\ep
    \end{prf}
    
    \begin{theorem}{Surjectivity $\Leftrightarrow$ Right-invertibility}
        Suppose $g: A\to B$ is a function. Then $g$ is injective if and only if there exists a function $h: B\to A$ such that $(g\circ h) = \text{id}_B$.
    \end{theorem}
    
    \begin{theorem}{\textbf{BIJECTIVITY $\Leftrightarrow$ INVERTIBILITY}}
        A function $f: A \to B$ is bijective $\Leftrightarrow$ $f$ has an inverse.
    \end{theorem}
    
    \clearpage
    
    \begin{theorem}{\textit{A countable union of (disjoint) countable sets is countable.}}
        For any collection $\left\{A_i: i \in I, |A_i| \le |\N|\right\}$ where $|I| \le |\N|$, we have $|\union_{i\in I} A_i| \le |\N|$.
        
        \begin{itemize}
            \item[\textbf{--}] "Countability squared" = "Countability"
        \end{itemize}
    \end{theorem}
    \begin{prf}{}
        We know $\exists f: I \xhookrightarrow{} \N$ which is injective, and $(\forall i \in I)[\exists g_i: A_i \xhookrightarrow{} \N]$ which is injective. \vspace{0.25cm}
        
        Define a map $$F: \bigcup_{i\in I} A_i \to \N, \hspace{0.25cm} a \mapsto 2^{f(n)}\cdot 3^{g_n(a)}, \hspace{0.25cm} \text{where $a \in A_n$.}$$
        
        Power of $2$ tracks the set. Power of $3$ tracks the element within the set. By the \textit{Fundamental Theorem of Arithmetic}, $F$ is truly injective. \ep
    \end{prf}
    
    \begin{theorem}{$|\N| = |\Z| = |\Q|$}
        Integers and rationals are countable.
    \end{theorem}
    \begin{prf}{}
        Proof if these facts is fairly simply.
        \begin{enumerate}
            \item $\N \subset \Z \subset \Q \Rightarrow |\N| \le |\Z| \le |\Q|$
            \item $\Z = \{-1, 0, 1\} \times \N \Rightarrow |\Z| \le |\N|$, according to the properties of a countable union of countable sets.
            \item $\Q = \N \times \Z \Rightarrow |\Q| \le |\N|$, according to the properties of a countable union of countable sets.
        \end{enumerate}
        Thus, integers and rationals are truly countable.
    \end{prf}
    
    \begin{theorem}{$|\R| > |\N|$}
        The set of real numbers is not countable.
    \end{theorem}
    
    \begin{prf}{"Gaussian Diagonalization"}
    This is just an instruction for the proof.
        \begin{enumerate}
            \item Prove that $|(0, 1)| = |\R|$.
            \item For the sake of contradiction, assume that $|(0, 1)| = |\N|$.
            \item "Invert" all digits on the diagonal, avoiding assigning the $base-1$ digit.
            \item The obtained number is real and will not match with any of our numbers. 
            \item Since we claimed to write down all real numbers this is a contradiction. 
            \ep
        \end{enumerate}
    \end{prf}

\end{document}
